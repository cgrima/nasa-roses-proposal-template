\section{DATA MANAGEMENT PLAN}

%\
%%%%%%%%%%%%%%%%%%%%%%%%%%%%%%%%%%%%%%%%%%%%%%%%%%%%%%%%%%%%%%%%%%%%%%%%%%%%%%%%%%%%%%%%%%%%%%%%%%%%%%%%%%%%%%%%%%%%%%%%%%%%%%%%%%%%%%%%%%%%%%%%%%%%%%%%%%%%%%%%%%%%%%%%\
%%%                                                                                                                                							%%%\
%%% Proposers should refer to the following documents in preparing their DMPs:                                               								%%%\
%%%                                                                                                                                							%%%\
%%%     1. ROSES program element C.1, Section 3.6.1                                                                                							%%%\
%%%     2. Planetary Science Division FAQ for Data Management Plans                                                                							%%%\
%%%        (which will appear under "other documents" on NSPIRES pages in Appendix C)                                              							%%%\
%%%     3. NASA Plan for Increasing Access to results of Federally funded research                                                 							%%%\
%%%        (https://smd-prod.s3.amazonaws.com/science-green/s3fs-public/atoms/files/NASA_Plan_for_increasing_access_to_results_of_federally_funded_research1.pdf),	%%%\
%%%		 Part A, Sections 1 \'96 6.    																										%%%\
%%%                                                                                                                                							%%%\
%%%%%%%%%%%%%%%%%%%%%%%%%%%%%%%%%%%%%%%%%%%%%%%%%%%%%%%%%%%%%%%%%%%%%%%%%%%%%%%%%%%%%%%%%%%%%%%%%%%%%%%%%%%%%%%%%%%%%%%%%%%%%%%%%%%%%%%%%%%%%%%%%%%%%%%%%%%%%%%%%%%%%%%%\

\subsection{Overview of the data that will be produced by the proposed project}
% (Describe project data needed to validate the scientific conclusions of\
% peer-reviewed publications - especially data underlying figures, maps,\
% and tables - as well as data that would enable future research and/or\
% the replication/reproduction of published results. If the project would\
% produce data that are exempted in the NASA Plan for Increasing Access to\
% the Results of Scientific Research, or no data that are scientifically\
% appropriate for public release, explain why.)\

<YOUR TEXT HERE>


\subsection{Data types, volume, formats, and, (where relevant), standards}
% (Describe the major types of data produced by the project [e.g., images,\
% 1-dimensional spectra, multidimensional tables]; the approximate amount\
% of each type expected [e.g., 300 1-dimensional spectra, each of ~10kB];\
% the format of the data, [e.g., FITS image files, ASCII tables, Excel\
% spreadsheets]; and any applicable standards for the data or metadata content\
% or format [e.g., PDS4, EarthChem].)\

<YOUR TEXT HERE>


\subsection{Schedule for data archiving and sharing}
% (Provide an anticipated schedule or timeline for when project data would\
% be prepared for and deposited in the repository and when they would become\
% publicly available. A timeline relative to the publication of major\
% results is acceptable. Please use project years and quarters rather than\
% calendar years and quarters.)\

<YOUR TEXT HERE>


\subsection{Intended repositories for archived data and mechanisms for public access and distribution}
% (State where the project data are intended to be archived, and describe\
% the terms under which data would be made available by the repository. \
% Repositories are expected to provide data without restriction or fees\
% other than the nominal costs of reproduction and shipping; i.e., they must\
% be publicly accessible with no paywall. If no appropriate repository exists,\
% please explain the situation and state what steps will be taken to provide\
% some degree of access.)\

<YOUR TEXT HERE>


\subsection{Plan for enabling long-term preservation of the data}
% (State how the intended repositories will preserve the data and provide\
% public access on a time-scale of one decade or longer.)\

<YOUR TEXT HERE>


\subsection{Software archiving plan}
% (Describe plans to archive any software required to enable future research\
% and/or the replication/reproduction of published results [see full\
% instructions in ROSES Appendix C.1]. Software should be made publicly\
% available when it is practical and feasible to do so and when there is\
% scientific utility in doing so. Any source code that is made publicly\
% available should be distributed, with appropriate documentation, via\
% GitHub, the PDS, or other appropriate community-recognized repository. \
% If software would be developed but not archived, explain why.)\

<YOUR TEXT HERE>


\subsection{Astromaterials archiving plan}
% (If your proposal includes plans to acquire or collect astromaterials, such as\
% meteorites, micrometeorites, or cosmic dust, describe plans to make publicly \
% available material not consumed during the research [see full instruction in ROSES\
% Appendix C.1].  Such astromaterials should be made available when it is practical \
% and feasible to do so and when there is scientific utility in doing so.  This section\
% may optionally cover how other physical materials collected, purchased, or synthesized \
% during the planned research would be made publicly available.)\

<YOUR TEXT HERE>


\subsection{Roles and responsibilities of team members for data management}
% (Explain which team members would perform data archiving tasks and\
% indicate explicitly what those tasks would be.  If there are costs\
% associated with data archiving, those must appear \'96 with explanation \'96\
% in the proposal budget.)\

<YOUR TEXT HERE>